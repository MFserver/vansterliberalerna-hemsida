\documentclass[a4paper,10pt]{article}

\usepackage{amssymb, graphicx, fancyhdr, multirow, tikz}
\usepackage[T1]{fontenc}
\usepackage[swedish]{babel}
\usepackage[utf8]{inputenc}

\setlength{\parindent}{0pt}
\setlength{\parskip}{12pt}
\pagestyle{fancy}
\renewcommand{\headrulewidth}{0.2pt}

\author{Togald Nilsson}
\title{\textbf{Manual för Vänsterliberalernas hemsida}}

\lhead{}
\rhead{}
\chead{}
\lfoot{}
\rfoot{}

\begin{document}

\maketitle
\tableofcontents
\pagebreak

\section{Inledning}
Vänsterliberalernas hemsida är inte som vilken hemsida som helst. Hemsidan skapas med en teknik som kan kallas developer-side PHP, vilket innebär att hemsidan genereras med hjälp av PHP av utvecklaren, innan den ens läggs upp på servern. Man kan säga att webbsidan måste kompileras innan den kan användas. PHP används av den anledningen att det är lätt att bädda in i HTML-koden. 

Eftersom webbsidan i dagsläget bara använder PHP på utvecklarsidan, behövs ingen server med PHP installerat för att köra webbsidan. Däremot måste utvecklaren ha PHP installerat på sin dator. Det är möjligt att använda PHP för att generera dynamiskt innehåll på webbsidan också, mer om det i sektion \ref{server-side-php}.

\section{Hur fungerar det?}
Vänsterliberalernas hemsida har tre mappar, {\tt content/}, {\tt source/} och {\tt target/}. 

\begin{itemize}
	\item {\tt content/} innehåller mallar till alla de sidor som ska genereras. 
	\item {\tt source/} innehåller alla källfiler, {\tt creator.php}, {\tt skel.php} och bilder, style sheets och så vidare som ska finnas på sidan. 
	\item {\tt target/} är vanligtvis tom vid start. Det är i denna mapp som webbsidan genereras. 
\end{itemize}

{\tt source/creator.php} är den fil där all magi sker. Det är den som används för att kompilera mallarna till en fungerande webbsida. Filen är väl kommenterad, och dess arbetsordning beskrivs här. 

\begin{enumerate}
	\item Leta igenom {\tt content/} efter mallar, och kompilera en webbsida för varje mall som hittas med {\tt source/skel.php} som skelett. 
	\item Placera mallfilen på motsvarande plats i {\tt target/}. Alla filer som inte är indexfiler kopieras utan behandling. 
	\item Kopiera över bilder, style sheets och andra statiska dokument från {\tt source/} till {\tt target/}.
\end{enumerate}

\subsection{Struktur i {\tt content/}}
För varje webbsida som ska skapas, måste en mapp med sidans namn skapas i {\tt content/}. I denna mapp skapas en fil {\tt index.php}, som är den fil som används av {\tt source/creator.php} för att generera webbsidan. Indexfilen behöver inte innehålla någon struktur alls, förutom en deklaration av PHP-variabeln {\tt \$title} någonstans i dokumentet, förslagsvis i början. {\tt \$title} används för att skriva ut sidans titel i titelraden i webbläsaren. 

Fördelarna med att använda mappar istället för HTML-sidor för att definiera strukturen, är framför allt att

\begin{itemize}
	\item URL-erna blir enklare, till exempel {\tt http://vl.mfserver.net/politik/} istället för {\tt http://vl.mfserver.net/politik.php}. Detta är viktigt framför allt för undersidor, som annars kan få mycket skumma URL-er. 
	\item det gör det lättare att skapa undersidor till webbsidorna. Istället för att skapa en separat mapp för undersidor till en viss sida, lägger man bara alltsammans i sidans mapp. 
\end{itemize}

\subsection{Arbetsflödet}
\begin{enumerate}
	\item Klona webbsidan från GitHub. Du borde veta hur man gör det. Tips: {\tt git clone}. Om du redan har webbsidan lokalt kan du hoppa över det här steget. 
	\item Redigera eller skapa de sidor du vill ha i {\tt content/}. Glöm inte att använda korrekt HTML5 och att deklarera PHP-variabeln {\tt \$title}. 
	\item Öppna {\tt source/skel.php}, och leta dig fram till sektionen som definierar sidmenyn. Om du har skapat eller redigerat sidor som du vill ska synas i menyn, måste du uppdatera den manuellt. 
	\item Öppna en terminal, och navigera till mappen dit du klonade hemsidan. 
	\item Öppna {\tt source/creator.php} och kontrollera att {\tt \$Root} är satt att peka till den mapp där du klonade webbsidan. Kontrollera även att du har kommenterat ut rätt deklaration av {\tt \$URL}. 
	\item Kör {\tt php source/creator.php} i terminalen. Du borde få en utskrift av alla filer som den har behandlat och kopierat. 
\end{enumerate}

\subsection{PHP-variabler du kan använda}
\begin{description}
	\item[{\tt \$title}] är webbsidans titel. Den kan med fördel användas även som rubrik, till exempel genom att skriva {\tt <h1><?php echo \$title; ?></h1>}
	\item[{\tt \$Target}] är den mapp som webbsidan genereras till. Även om denna variabel inte förefaller så användbar, så är den tillgänglig för indexfilen. Troligen bör du använda {\tt \$URL} istället i de flesta fall. 
	\item[{\tt \$URL}] är webbsidans URL-address. Denna är mycket användbar när du ska länka. Istället för att skriva {\tt <a href=''http://vansterliberalerna.se/sida/index.php''>...</a>}, så kan man skriva {\tt <a href=''<?php echo \$URL; ?>sida/index.php''>...</a>}. Det kanske inte ser ut som någon förenkling, men det för med sig en sak som är en enorm fördel: när webbsidan flyttas, uppdateras URL-er automatiskt. Alla som någon gång har flyttat en webbsida med vanliga hårdlänkar vet att det är ett helvete att byta ut dem manuellt. 
\end{description}

\subsection{Att använda PHP dynamiskt på servern}
\label{server-side-php}
Som tidigare nämnts, är det fullt möjligt att använda PHP både för att kompilera webbsidan, och för att generera dynamiskt innehåll. Det lättaste är att helt enkelt använda olika öppningstaggar för PHP beroende på om det som ska genereras är dynamiskt innehåll eller statiskt innehåll. 

Som standard använder PHP {\tt <?php ... ?>} för att markera ett block med PHP-kod. Men det finns två andra varianter: 

\begin{description}
	\item[{\tt <? ... ?>}] är kortformen av standardtaggen. 
	\item[{\tt <\% ... \%>}] är den tagg som används av Microsoft ASP, och är tillgänglig även i PHP. 
\end{description}

PHP avråder från att man använder {\tt <? ... ?>}, eftersom det kan skapa problem med XML om man använder det. Vänsterliberalernas webbsida gör inte det, men vi föreslår ändå att {\tt <?php ... ?>} används för statiskt innehåll (developer-side), {\tt <\% ... \%>} för dynamiskt innehåll (server-side), och {\tt <? ... ?>} inte alls. 

\subsection{Bilder}
Vänsterliberalerna har två sätt att inkludera bilder på webbsidan. Delade bilder, bilder som ska finnas tillgängliga från alla sidor på hela hemsidan, ska läggas direkt i {\tt source/img/}. Denna mapp ska inte innehålla fler bilder än att de kan lagras i en enda mapp utan att det blir plottrigt. Exempel på bilder som ligger här är ikoner för sociala media, partiets logga, thumbnails av partistyrelsens medlemmar. 

Bilder som inte ska delas mellan flera sidor, ska lagras i samma mapp som indexdokumentet för sidan de ska användas på. Till exempel pressbilder lagras direkt i {\tt content/press/}, fullstora bilder på styrelsemedlemmar lagras i medlemmens mapp i {\tt content/styrelsen/}. 

Alla thumbnails skall vara kvadratiska, 96x96 pixlar. Fullstora bilder ska ha filnamnstillägget skrivet med tre bokstäver, {\tt .jpg}, {\tt .png}, {\tt .gif} och {\tt .svg} rekommenderas. Thumbnails ska ha suffixet {\tt .thumb}. 

De fullstora bilderna på styrelsemedlemmar ska ha bredden 200 pixlar, och får vara av valfri höjd. Det rekommenderas att bilden är i porträttläge, det vill säga högre än den är bred, men det är inte ett krav. 

\section{Tips}
Varje gång webbidan kompileras, skrivs hela {\tt target/} över med nya filer, oavsett om något ändrats eller inte. Om du har en SSD i datorn, är det inte optimalt. Då kan det vara bra att skapa ett litet utrymme på systemminnet, som du kan använda att jobba på. Sök på ''create a ramdisk'' på internet så hittar du garanterat något användbart. 

Du kan naturligtvis skapa ett PHP-skript för att köra {\tt source/creator.php} direkt på webbhotellet om du inte har möjlighet att installera PHP på din egen dator. Men då är det ganska svårt att generera dynamiskt innehåll på samma webbhotell. Totalt är det förmodligen lättare för dig att installera PHP på din egen dator. 

\end{document}